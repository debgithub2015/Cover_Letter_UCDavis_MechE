\documentclass{portfolio}


\geometry{margin=1.0in,top=0.5in,bottom=0.5in,ignoreheadfoot}
\pagestyle{empty}
\setlength{\parindent}{0in}
\setlength{\parskip}{0.15in}

\usepackage{lastpage}
\begin{document}
\letterhead

\vspace{0.02in}
To,\\
The Chair and the Search Committee,\\ 
Mechanical & Aerospace Engineering Department\\
University of California, Davis, California, USA\hfill \today
\vspace{0.02in}

Dear Sir/Madam,

\ifResearch{
%
I wish to apply for the recently advertised `Unit 18 Temporary Lecturers Pool' (Job \# JPF03810) in the Mechanical & Aerospace Engineering Department at UCDavis, CA to teach `Thermodynamics' (ENG 105), `Computer Programming for Engineering Applications' (EME 005), and `Introduction to Numerical Analysis & Methods' (EME 115) . My research background is theoretical and computational method development within the density functional theory (DFT). I am currently a Research Assistant Professor at Wake Forest University, NC, USA. I have done my Ph.D. from McMaster University, Canada, on the development of DFT for two-particle and beyond density functionals. I have done my Master's from IIT, Kharagpur, India in Chemistry and Bachelor's from Presidency College, Kolkata, India in Chemistry.

I have a strong research background in the development and application of Density Functional Theory (DFT). With my current position, I am proposing exciting new avenues of describing van der Waals interactions in chemical compounds by improving non-local correlations within van der Waals density-functional (vdW-DF) framework. My current activity can be seen here: \href{https://thonhauser.physics.wfu.edu/Main/HomePage}{https://thonhauser.physics.wfu.edu}. My earlier works on a) bench-marking two-particle kinetic energy densities, b) developments on orbital-free free energy functionals (VT84F), and c) development of four-element classical potential (Ni-Fe-Cr-Pd) are some of my other remarkable achievements.

In addition to research, I have substantial teaching experience. During my Ph.D. tenure at McMaster University, I taught Thermodynamics and Statistics, Physical Chemistry, and Theoretical and Computational Chemistry classes and conducted first-year undergraduate Chemistry laboratory classes. During my post-doctoral career, I have guided graduate and undergraduate thesis projects and given informal classes/hand-on-sessions on material science and solid-state codes to the Graduate students at the Mechanical Engineering Department at University of Wyoming. Though, my education is condensed matter and theoretical application towards Material Science. Hence, I am confident that I can undertake the Thermodynamics, Numerical Analysis, and Computer Program teaching course at introductory and advanced level in online or face-to-face classes and any other Theoretical and Computational Material Science/Solid-State Physics/Computational Chemistry teaching responsibility in your department, as required.  }

\ifTeaching{
%
%ADD if you would like to go for teaching position!!
}
%\newpage
%\letterhead

%My objective, as a teacher, is in general to motivate my students to develop their own learning interests and critical thinking to establish a learning-centered environment in the classroom.  I usually prepare lessons considering the student’s own knowledge, learning abilities and the subject matter; choose the content and activities in such a way that students can think “beyond” their comfort zone and involve in their own
%learning process. I believe that clear, open communication with students is a key element in helping them learn and I would greatly value the supportive and caring environment between professors and students within the department.

Given my research interests, teaching experiences, and postdoctoral research background, I believe I would
fit well in your department and contribute to the departmental initiatives on  high-class teaching, however, I do not have any teaching evaluation documents to show.  I am very excited about this opportunity at the University. Please find the enclosed
requested application materials. I eagerly look forward to hearing from you soon.


Sincerely, \\
Debajit Chakraborty,\\
\textit{Physics, Wake Forest University}

\end{document}
